\documentclass[10pt, twocolumn]{article}

\usepackage{polski}
%\usepackage[T1]{fontenc}
%\usepackage[polish]{babel}
\usepackage[latin2]{inputenc}
\usepackage{indentfirst}
\usepackage{amsmath}
\usepackage{amssymb}
\usepackage{latexsym}
\usepackage{array}
\usepackage[top=2cm,left=2cm,right=2cm,,bottom=3cm,nohead]{geometry}
%\usepackage{nopageno}
%\pagestyle{empty}
\usepackage{graphicx}
\usepackage{subfig}
\usepackage{rotating}
\renewcommand{\abstractname}{Abstrakt}

\usepackage[sc, center]{titlesec}

\input{epsfx}

\title{\textsc{II Pracownia Fizyczna: \textbf{$\mathbf{\gamma}$1}\\Pomiar i analiza widm promieniowania gamma za pomocą spektrometru scyntylacyjnego. Wyznaczenie współczynnika absorbcji promieniowania gamma w metalach.}}
\author{Michał Rawlik\\
\small Uniwersytet Jagielloński\\
\small Wydział Fizyki, Astronomii i Informatyki Stosowanej}

\begin{document}

\twocolumn[
\begin{@twocolumnfalse}
  \maketitle
  \begin{abstract}
\noindent W ramach ćwiczenia zarejestrowano widma $^{133}\mathrm{Ba}$, $^{60}\mathrm{Co}$, $^{137}\mathrm{Cs}$ oraz $^{22}\mathrm{Na}$. Przeprowadzona kalibracja pozwoliła potwierdzić istnienie krawędzi Comptonowskich w miejscach przewidzianych przez teorię dla wszystkich pierwiastków poza barem. W drugiej części ćwiczenia zmierzono masowe współczynniki absorpcji dla ołowiu i aluminium otrzymując odpowiednio 0.09252(64) i 0.06950(27) cm$^2\,$g$^{-1}$; wyniki zgadzają się z wartościami podanymi w instrukcji~\cite{prac}.
  \end{abstract}
  \hspace{5cm}
\end{@twocolumnfalse}
]

\section*{Wstęp teoretyczny}

\subsection*{Emisja promieniowania gamma}
W wyniku rozpadu jądra atomowego nowo powstałe jądro może znaleźć się w stanie wzbudzonym po czym zdeekscytować emitując przy tym energie w postaci kwantu gamma o energii równej różnicy poziomów. Dodatkowo jeśli mamy do czynienia z rozpadem $\beta^+$ (np. w przypadku sodu $^{22}$Na) kwanty gamma emitowane są w wyniku anihilacji powstałego pozytonu z elektronem. Wtedy emitowane są fotony o łącznej energii równej energii promieniowania $\beta^+$ plus masa spoczynkowa elektronu (razy prędkość światła do kwadratu). Schematy rozpadu badanych izotopów (poza $^{133}\mathbf{Ba}$) przedstawiono na rys. \ref{schematy}.
\begin{figure}
  \center{\includegraphics[width=8cm]{schematy.pdf}}
  \caption{\em Schematy rozpadów badanych izotopów.}
  \label{schematy}
\end{figure}

\subsection*{Detekcja promieniowania gamma}
Do detekcji promieniowania gamma używa się \textbf{kryształów scyntylacyjnych}. Są one materiałami które emitują światło w zakresie widzialnym (błyski) pod wpływem promieniowania jonizującego przy czym natężenie błysku jest proporcjonalne do energii zdeponowanej w krysztale. Istotne jest, że energia kwantów gamma \textbf{nie jest mierzona bezpośrednio} -- bezpośrednio mierzona jest energia elektronów.

Kwanty gamma oddziałują z materią na trzy główne sposoby:
\begin{description}
   \item[efekt fotoelektryczny] który polega na przekazaniu całej energii fotonu związanemu elektronowi i tym samym uwolnienie go z pola jądra. Tym samym cała energia kwantu gamma jest przekazywana elektronowi. Proces ten odpowiada za powstawanie w mierzonym widmie \textbf{fotopiku} czyli zwiększonej ilości zdarzeń w kanałach odpowiadających rzeczywistej energii promieniowania
   \item[efekt Comptona] czyli przekazanie elektronowi przez foton jedynie części swojej energii. Po zderzeniu foton zmienia kierunek lotu i długość fali. Ponieważ elektronowi jest przekazywana tyko część energii to zjawisko odpowiada za tzw. ciągłe \textbf{tło komptonowskie} wyraźnie widoczne na widmie i kończące się przy energii odpowiadającej rozproszeniu fotonu do tyłu, tj. $E_\gamma - 256$keV \cite{agh} (co wynika z kinematyki reakcji po podstawieniu masy elektronu). Jednocześnie obserwowane będą \textbf{fotony} które rozproszyły się po comptonowsku do tyłu (w próbce lub detektorze) i wywołały efekt fotoelektryczny -- tworzą one tak zwany \textbf{pik rozproszenia wstecznego}.
   \item[tworzenie par] -- jeśli energia fotonu przekracza dwie masy spoczynkowe elektronu to w polu jądra może się wytworzyć para elektron-pozyton, które dalej są rejestrowane, przy czym jeden z nich może z ,,uciec'' detektora.
\end{description}

Błyski scyntylatora rejestrowane są przez fotopowielacz. Warto również wspomnieć, że wydajność takiego układu detekcyjnego zmienia się wraz z energią -- im niższa energia tym lepiej rejestrowane są zdarzenia.

Układ detekcyjny posiada pewną \textbf{energetyczną zdolność rozdzielczą}. Objawia się to tym, że gdy do detektora wpada idealnie monochromatyczne promieniowanie obserwowane widmo ma kształt zbliżony do rozkładu Gaussa. Miarą zdolności rozdzielczej może być \textbf{bezwzględna energetyczna zdolność rozdzielcza} równa szerokości połówkowej widma jakie zwraca detektor dla monochromatycznego promieniowania. \textbf{Względna energetyczna zdolność rozdzielcza} to stosunek szerokości połówkowej do położenia maksimum.

\subsection*{Absorpcja promieniowania gamma}
Ilość fotonów zaabsorbowanych na jednostkę drogi jest proporcjonalna do ich strumienia: $\mathrm{d}I = \sigma I \mathrm{d}x$ \cite{agh}. Rozwiązaniem tego równania jest prawo eksponencjalnego zaniku:
\begin{equation}
  \label{eq:absorbcja}
  I(x) = I_0 e^{-\sigma x}
\end{equation}
gdzie $\sigma$ to \textbf{liniowy współczynnik osłabienia}. Podzielony przez gęstość materiału daje \textbf{masowy współczynnik osłabienia}.

\subsection*{Prawa statystyczne}
Ilość rozpadów w próbce promieniotwórczej w określonym przedziale czasowym opisywana jest (przy założeniu, że czas pomiaru jest dużo krótszy od czasu połowicznego rozpadu) \textbf{rozkładem Poissona} \cite{agh}. Rozkład ten ma tylko jeden parametr: średnią liczbę zdarzeń $\bar{N}$. Odchylenie standardowe tego rozkładu wynosi $\sqrt{\bar{N}}$, przy czym dla pojedynczego pomiaru można je szacować przez $\sqrt{N}$. Dla dużej liczby zliczeń rozkład Poissona dąży do rozkładu Gaussa.

\section*{Opis pomiarów}
\subsubsection*{Dzień 1 -- 2 III 2012}
Pierwszego dnia poza kolokwium i zapoznaniem się z układem pomiarowym zmierzono widma czterech pierwiastków: $^{133}\mathbf{Ba}$, $^{60}\mathbf{Co}$,  $^{137}\mathbf{Cs}$ oraz $^{2}\mathbf{Na}$.

Tego dnia przeprowadzono także pomiary absorpcji promieniowania gamma w ołowiu. W tym celu pomiędzy próbką ($^{137}\mathbf{Cs}$) a detektorem umieszczono ołowiane płytki. Przeprowadzono serię 10 pomiarów po 720 sekund po każdym zdejmując jedną płytkę. Na koniec zmierzono promieniowanie tła (również 720 sekund).

\subsubsection*{Dzień 2 -- 7 III 2012}
Cały drugi dzień poświęcony został na pomiar absorpcji promieniowania gamma w aluminium. Pomiar odbywał się na takiej samej zasadzie jak dla ołowiu, z tą różnicą że czasami zdejmowano więcej płytek.



\section*{Opracowanie pomiarów}

\subsection*{Kalibracja}
Do widm $^{133}\mathbf{Ba}$ (rys. \ref{fit_ba}), $^{60}\mathbf{Co}$ (rys. \ref{fit_co}) oraz $^{137}\mathbf{Cs}$ (rys. \ref{fit_cs}) dopasowano przy pomocy programu \texttt{fityk} \cite{fityk} liniowe tło oraz funkcje Gaussa w miejscach pików.

Korzystając z tablicowych \cite{NuclearDataCenter} energii promieniowania gamma poszczególnych pierwiastków przeprowadzono kalibrację przyporządkowując każdemu kanałowi energię. Widma po skalibrowaniu przedstawiono na rys. \ref{widma}, a krzywą kalibracyjną na rys. \ref{kalibracja}.

\begin{figure}
  \center{\includegraphics[width=8.5cm]{widma.pdf}}
  \caption{\em Zmierzone widma promieniowania gamma emitowanego przez badanie pierwiastki.}
  \label{widma}
\end{figure}

\subsection*{Analiza widm}
Na rysunkach \ref{ba133}-\ref{na22} przedstawiono poszczególne widma wraz z opisami. Pionowymi liniami oznaczono tam tablicowe energie promieniowania, jak również obliczone krawędzie Comptonowskie. Pików jakie powinny zestawiać elektrony po rozpraszaniu wstecznym nie udało się zaobserwować. Zgodnie z teorią powinny się znajdować w każdym widmie (o ile energia fotonów na to pozwala) przy energii 256$\,$keV. Na rys. \ref{widma} widać, że przy tej energii nie występują żadne widoczne piki na widmach.

\subsection*{Zdolność rozdzielcza}
Korzystając z widma $^{137}\mathbf{Cs}$ (rys. \ref{cs137}) i dopasowanej krzywej (rys. \ref{fit_cs}) wyznaczono energetyczną zdolność rozdzielczą układu detekcyjnego. Otrzymano:
\begin{equation*}
  \begin{array}{ll}
    \textbf{bezwzględna} & \mathbf{85\,keV} \\
    \textbf{względna} & \mathbf{12.5 \%}
  \end{array}
\end{equation*}


\subsection*{Absorpcja}
Absorpcję badano umieszczając między źródłem ($^{137}\mathbf{Cs}$) a detektorem szereg płytek ołowianych oraz aluminiowych. Każdy z pomiarów trwał 720 sekund. Wszystkie zmierzone widma przedstawiono na rys. \ref{wszystkie_al} i \ref{wszystkie_pb}.

Od wszystkich widm odjęto tło a następnie do fotopiku na każdym z nich dopasowano krzywą Gaussa i obliczono pole pod nią ($N_{max}2 \pi \sigma$). Do zależności logarytmu obliczonego pola od grubości warstwy metalu dopasowano prostą (rys. \ref{ab_pb} i \ref{ab_al}). Otrzymane liniowe współczynniki osłabienia:
\begin{equation*}
  \begin{array}{ll}
    \text{ołów} & 0.1052(73)\,\text{mm}^{-1} \\
    \text{aluminium} & 0.018765(73)\,\text{mm}^{-1}
  \end{array}
\end{equation*}
Uwzględnienio gęstości metali ($\rho_{Pb} = 11.37\ \mathrm{g\,cm}^{-3}$, $\rho_{Al} = 2.7\ \mathrm{g\,cm}^{-3}$ pozwala obliczyć masowe współczynniki osłabienia:
\begin{equation*}
  \begin{array}{ll}
    \textbf{ołów} & \mathbf{0.09252(64)\,cm^2\,g^{-1}} \\
    \textbf{aluminium} & \mathbf{0.06950(27)\,cm^2\,g^{-1}}
  \end{array}
\end{equation*}

Tablicowe wartości \cite{prac} wynoszą odpowiednio $\approx 0.09$ i $\approx 0.08$, co dobrze zgadza się z uzyskanymi wynikami.

\begin{figure}
  \center{\includegraphics[width=8.5cm]{olow.pdf}}
  \caption{\em Ilość zliczeń w fotopiku w zależności od grubości warstwy ołowiu pomiędzy źródłem a detektorem.}
  \label{ab_pb}
\end{figure}

\begin{figure}
  \center{\includegraphics[width=8.5cm]{aluminium.pdf}}
  \caption{\em Ilość zliczeń w fotopiku w zależności od grubości warstwy aluminium pomiędzy źródłem a detektorem.}
  \label{ab_al}
\end{figure}


\section*{Podsumowanie}
Udało się zebrać widma czterech pierwiastków i skalibrować układ doświadczalny. Jak widać na wykresach widm kalibracja niestety nie jest zbyt dokładna (kilka pików nie ma maksimów w energiach określonych z teorii). Na widmach udało się zidentyfikować krawędzie comptonowskie, aczkolwiek już piki rozproszenia wstecznego były niewidoczne. Dodatkowo wyznaczono rozdzielczość energetyczną układu doświadczalnego.

Z kolei część ćwiczenia polegającą na wyznaczeniu współczynników absorpcji można uznać za w pełni udaną. Zostały one wyznaczone z dużą dokładnością (rzędu promili) i zgadzają się z danymi zamieszczonymi w \cite{prac}.


\begin{thebibliography}{00}
\bibitem{agh}
B. Dziunikowski, S. J. Kalita \emph{Ćwiczenia laboratoryjne z jądrowych metod pomiarowych}

\bibitem{fityk}
program \emph{fityk} \verb|http://fityk.nieto.pl/|

\bibitem{NuclearDataCenter}
\emph{Nuclear Data Center of Japan Atomic Energy Agency} \verb|http://wwwndc.jaea.go.jp/|

\bibitem{wiki}
Wikimedia Commons \\ \verb|http://commons.wikimedia.org/|

\bibitem{prac}
\emph{Instrukcja do ćwiczenia $\gamma1$} \\\verb|http://users.uj.edu.pl/~pracfiz2/PDF/G1.pdf|

\bibitem{wa}
Wolfram Alpha \\ \verb|http://www.wolframalpha.com/|
\end{thebibliography}

%OBRAZKI

\begin{figure}
  \center{\includegraphics[width=8.5cm]{fit_ba133.pdf}}
  \caption{\em Dopasowanie krzywych Gaussa i tła do widma $^{133}\mathbf{Ba}$. Środki pików (i ich wysokości), od lewej: 76 (203), 102 (1249), 189 (944), 226 (2051). }
  \label{fit_ba}
\end{figure}
\begin{figure}
  \center{\includegraphics[width=8.5cm]{fit_co60.pdf}}
  \caption{\em Dopasowanie krzywych Gaussa i tła do widma $^{60}\mathbf{Co}$. Środki pików (i ich wysokości), od lewej: 681 (5487), 768 (3588). }
  \label{fit_co}
\end{figure}
\begin{figure}
  \center{\includegraphics[width=8.5cm]{fit_cs137.pdf}}
  \caption{\em Dopasowanie krzywej Gaussa i tła do widma $^{137}\mathbf{Cs}$. Środek piku znajduje się w położeniu 407, a jego wysokość to 3537.}
  \label{fit_cs}
\end{figure}

\begin{figure}
  \center{\includegraphics[width=8.5cm]{kalibracja.pdf}}
  \caption{\em Krzywa kalibracyjna znaleziona na podstawie porównania położeń fotopików w zmierzonych widmach z tablicowymi \cite{NuclearDataCenter} wartościami energii promieniowania $^{133}\mathbf{Ba}$, $^{60}\mathbf{Co}$ oraz $^{137}\mathbf{Cs}$.}
  \label{kalibracja}
\end{figure}

\begin{figure}
  \center{\includegraphics[width=8.5cm]{ba133.pdf}}
  \caption{\em Zmierzone widmo baru. Zaznaczono na nim położenie dwóch nakładających się na siebie fotopików. Na tym widmie niewidoczne są krawędzie comtonowskie, najprawdopodobniej ze względu na niską energię fotonów.}
  \label{ba133}
\end{figure}

\begin{figure}
  \center{\includegraphics[width=8.5cm]{co60.pdf}}
  \caption{\em Zmierzone widmo kobaltu. Zaznaczono oba fotopiki pochodzące od deekscytacji jądra wraz z odpowiadającymi im krawędziami comptonowskimi. Wyraźna jest jedynie krawędź pochodząca od piku przy mniejszej energii, druga prawdopodobnie ,,ginie'' w piku.}
  \label{co60}
\end{figure}

\begin{figure}
  \center{\includegraphics[width=8.5cm]{cs137.pdf}}
  \caption{\em Zmierzone widmo cezu. Zaznaczono fotopik pochodzący z deekscytacji jądra oraz odpowiadającą mu krawędź comptonowską. Na tym przykładzie wyraźna jest niedokładność kalibracji.}
  \label{cs137}
\end{figure}

\begin{figure}
  \center{\includegraphics[width=8.5cm]{na22.pdf}}
  \caption{\em Zmierzone widmo sodu. Widać fotopik pochodzący z deekscytacji jądra oraz pik pochodzący z anihilacji pozytonu z rozpadu $\beta^+$. Emitowany pozyton ma energię maksymalną 545$\,$keV, anihiluje z elektronem o masie 511$\,$keV emitując dwa fotony, każdy o energii 528$\,$keV. Zaznaczone są również krawędzie comptonowskie odpowiadające obu rodzajom fotonów.}
  \label{na22}
\end{figure}

\begin{figure}
  \center{\includegraphics[width=8.5cm]{wszystkie_al.pdf}}
  \caption{\em Wszystkie zmierzone widma podczas pomiarów absorpcji aluminium oraz pomiar tła.}
  \label{wszystkie_al}
\end{figure}

\begin{figure}
  \center{\includegraphics[width=8.5cm]{wszystkie_pb.pdf}}
  \caption{\em Wszystkie zmierzone widma podczas pomiarów absorpcji ołowiu oraz pomiar tła.}
  \label{wszystkie_pb}
\end{figure}


\end{document}
